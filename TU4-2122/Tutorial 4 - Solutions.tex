\documentclass[11pt,a4paper]{article}
%\documentclass[professionalfonts, 10pt]{beamer}
%\usetheme{default}
%\usetheme{Pittsburgh}
\usepackage{accessibility}
\usepackage[a4paper, total={6in, 8in}]{geometry}
\usepackage[pdflang={en-GB}]{hyperref}
\renewcommand{\familydefault}{\sfdefault}
%
%
\usepackage{enumitem}
\usepackage[T1]{fontenc}
\usepackage{hyperref}
\usepackage{kerkis}
\usepackage{stmaryrd}
%\usepackage{bm}
%\usepackage{mathrsfs}
\usepackage{bm}
\usepackage{amsmath,amssymb,amsfonts,amsthm,mathrsfs,dsfont}
%\usepackage{latexsym}
%\usefonttheme{professionalfonts}
%\usetheme{Copenhagen}
%\usecolortheme{beaver}
\usepackage[english]{babel}
\usepackage{textcomp}
\usepackage{marvosym}
\usepackage[latin1]{inputenc}
\usepackage{times}
\usepackage{textcomp}
%\usepackage{pifont}
%\usepackage{bbding}
%\usepackage{enumerate}
\usepackage{natbib}
%\usepackage{pgf,pgfarrows,pgfnodes,pgfautomata,pgfheaps}
\usepackage{color}
\usepackage{graphicx}
%
%
%\usepackage{authblk}
%\usepackage{MnSymbol}
\usepackage{semtrans}
\usepackage{units}
%\usepackage{mathrsfs}
%\usepackage{makeidx}
%
%
\newcommand\independent{\protect\mathpalette{\protect\independenT}{\perp}}

\def\independenT#1#2{\mathrel{\rlap{$#1#2$}\mkern2mu{#1#2}}}
\setcounter{MaxMatrixCols}{10}
\newcommand{\stackunder}[2]{\mathrel{\mathop{#2}\limits_{#1}}}
\newcommand{\fd}[1]{\Delta^{#1}_+}
%
\DeclareMathOperator{\E}{\mathbb{E}}
\DeclareMathOperator{\R}{\mathbb{R}}
\DeclareMathOperator{\Z}{\mathbb{Z}}
\DeclareMathOperator{\N}{\mathbb{N}}
\DeclareMathOperator{\var}{\mathbb{V}ar}
\DeclareMathOperator{\cov}{\mathbb{C}ov}
\DeclareMathOperator{\corr}{\mathbb{C}orr}
%
\def\vec{\mathop{\textmd{vec}}\nolimits}
\def\eig{\mathop{\textmd{eig}}\nolimits}
\def\tr{\mathop{\textmd{tr}}\nolimits}
\def\diag{\mathop{\textmd{diag}}\nolimits}
\def\vec{\mathop{\textmd{vec}}\nolimits}
\def\vh{\mathop{\textmd{vech}}\nolimits}
\def\berr{\begin{eqnarray*}}
\def\eerr{\end{eqnarray*}}
\def\ber{\begin{eqnarray}}
\def\eer{\end{eqnarray}}
\def\d{\textrm{d}}
\def\corr{\mathop{\textmd{Corr}}\nolimits}
%%%%%%%%%%%%%%%%%%%%%%%%%%%%%%%%%%%%%%%%%%%%%%%%%%%%%%%%%%%%%%%%%%%%%%%%%%%%%%%%%%
%%%%%%%%%%%%%%%%%%%%%%%%%%%%%%%%%%%%%%%%%%%%%%%%%%%%%%%%%%%%%%%%%%%%%%%%%%%%%%%%%%


\def\ln{\textrm{ln}}

\def\itemb{\item[$\bullet$]}
%%%%%%%%%%%%%%%%%%%%%%%%%%%%%%%%%%%%%%%%%%%%%%%%%%%%%%%%%%%%%%%%%%%%%%%%%%%%%%%%%%%%%
%%%%%%%%%%%%%%%%%%%%%%%%%%%%%%%%%%%%%%%%%%%%%%%%%%%%%%%%%%%%%%%%%%%%%%%%%%%%%%%%%%%%%






\newcommand{\V}[1]{``{#1}"}

\begin{document}
\begin{center}
\begin{Huge}
Tutorial 4
\end{Huge}
\end{center}


\begin{enumerate}[labelindent=0pt,labelwidth=0.75em,leftmargin=!]
\item  As part of a \V{Math for the Twenty-First Century} initiative, Bayview High  was chosen to participate in the evaluation of a new algebra and geometry curriculum. In the  recent past, Bayview's students were considered \V{typical}, having earned scores on standardized exams that were very consistent with national averages.

Two years ago, a cohort of eighty-six Bayview students, all randomly selected, were assigned to a special set of classes that integrated algebra and geometry. According to test results that have just been released, those students averaged 502 on the math exam; nationwide seniors averaged 494 with a standard deviation of 124.
\begin{enumerate}
\item Can it be claimed at $5\%$ significance level that the new curriculum had effect? Justify your answer.
\item Compute the p-value associated with the test statistics. How should it be interpreted? 
\end{enumerate}

\item[]\textbf{Short Solution}
\begin{enumerate}
\item Given the sample size and the sampling scheme, the sample average is asymptotically normally distributed. 
 $\frac{502-494}{124/\sqrt{n}}=0.5983$ is lower than 1.96, so we don't reject the null (even at 10\% significance level).
\item Looking at the statistical tables
$$
P\left(Z\leq 0.5983\right)\simeq 0.7257
$$
Then the p-value is $2(1-0.7257)=0.5486$. The large p-value should be interpreted as evidence against the rejection of the null hypothesis. We would reject the null hypothesis for unusual large $\alpha>0.5486$.
\end{enumerate}
\item Supporters claim that a new windmill can generate
an average of at least 800 kilowatts of power per day.
Daily power generation for the windmill is assumed
to be normally distributed with a standard deviation of 120 kilowatts. A simple random sample of 100 days is
taken to test this claim against the alternative hypothesis that the true mean is less than 800 kilowatts. The
claim will not be rejected if the sample mean is 776
kilowatts or more and rejected otherwise. 
\begin{enumerate}
\item What is the probability $\alpha$ of a Type I error using
the decision rule if the population mean is, in fact,
800 kilowatts per day?
\item What is the probability $\beta$ of a Type II error using
this decision rule if the population mean is, in fact,
740 kilowatts per day?
\item Suppose that the same decision rule is used, but
with a sample of 200 days rather than 100 days.
\begin{enumerate}
\item Would the value of $\alpha$ be larger than, smaller
than, or the same as that found in part (a)? Explain.
\item Would the value of $\beta$ be larger than, smaller
than, or the same as that found in part (b)? Explain.
\end{enumerate}
\end{enumerate}
%
\item[]\textbf{Short Solution} See the solution of Exercise 9.58 Textbook (see Moodle, Unit 5).

\end{enumerate}
\end{document}