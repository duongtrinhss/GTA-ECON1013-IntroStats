% Options for packages loaded elsewhere
\PassOptionsToPackage{unicode}{hyperref}
\PassOptionsToPackage{hyphens}{url}
%
\documentclass[
  10pt,
  ignorenonframetext,
]{beamer}
\usepackage{pgfpages}
\setbeamertemplate{caption}[numbered]
\setbeamertemplate{caption label separator}{: }
\setbeamercolor{caption name}{fg=normal text.fg}
\beamertemplatenavigationsymbolsempty
% Prevent slide breaks in the middle of a paragraph
\widowpenalties 1 10000
\raggedbottom
\setbeamertemplate{part page}{
  \centering
  \begin{beamercolorbox}[sep=16pt,center]{part title}
    \usebeamerfont{part title}\insertpart\par
  \end{beamercolorbox}
}
\setbeamertemplate{section page}{
  \centering
  \begin{beamercolorbox}[sep=12pt,center]{part title}
    \usebeamerfont{section title}\insertsection\par
  \end{beamercolorbox}
}
\setbeamertemplate{subsection page}{
  \centering
  \begin{beamercolorbox}[sep=8pt,center]{part title}
    \usebeamerfont{subsection title}\insertsubsection\par
  \end{beamercolorbox}
}
\AtBeginPart{
  \frame{\partpage}
}
\AtBeginSection{
  \ifbibliography
  \else
    \frame{\sectionpage}
  \fi
}
\AtBeginSubsection{
  \frame{\subsectionpage}
}
\usepackage{amsmath,amssymb}
\usepackage{lmodern}
\usepackage{iftex}
\ifPDFTeX
  \usepackage[T1]{fontenc}
  \usepackage[utf8]{inputenc}
  \usepackage{textcomp} % provide euro and other symbols
\else % if luatex or xetex
  \usepackage{unicode-math}
  \defaultfontfeatures{Scale=MatchLowercase}
  \defaultfontfeatures[\rmfamily]{Ligatures=TeX,Scale=1}
\fi
% Use upquote if available, for straight quotes in verbatim environments
\IfFileExists{upquote.sty}{\usepackage{upquote}}{}
\IfFileExists{microtype.sty}{% use microtype if available
  \usepackage[]{microtype}
  \UseMicrotypeSet[protrusion]{basicmath} % disable protrusion for tt fonts
}{}
\makeatletter
\@ifundefined{KOMAClassName}{% if non-KOMA class
  \IfFileExists{parskip.sty}{%
    \usepackage{parskip}
  }{% else
    \setlength{\parindent}{0pt}
    \setlength{\parskip}{6pt plus 2pt minus 1pt}}
}{% if KOMA class
  \KOMAoptions{parskip=half}}
\makeatother
\usepackage{xcolor}
\newif\ifbibliography
\setlength{\emergencystretch}{3em} % prevent overfull lines
\providecommand{\tightlist}{%
  \setlength{\itemsep}{0pt}\setlength{\parskip}{0pt}}
\setcounter{secnumdepth}{-\maxdimen} % remove section numbering
\ifLuaTeX
  \usepackage{selnolig}  % disable illegal ligatures
\fi
\IfFileExists{bookmark.sty}{\usepackage{bookmark}}{\usepackage{hyperref}}
\IfFileExists{xurl.sty}{\usepackage{xurl}}{} % add URL line breaks if available
\urlstyle{same} % disable monospaced font for URLs
\hypersetup{
  pdftitle={Introductory Statistics for Economics},
  pdfauthor={Duong Trinh},
  hidelinks,
  pdfcreator={LaTeX via pandoc}}

\title{Introductory Statistics for Economics}
\subtitle{ECON1013: TUTORIAL 3}
\author{Duong Trinh}
\date{Feb 2022}
\institute{University of Glasgow}

\begin{document}
\frame{\titlepage}

\begin{frame}{Exercise 1}
\protect\hypertarget{exercise-1}{}
In the mid-term elections of 2006, the political winds were shifting.
One of the key races for control of the Senate was in Virginia, where
challenger Jim Webb and incumbent George Allen were in a very tight
race. Just a week before the election, the Associated Press reported on
a CNN poll based on telephone interviews of \(597\) registered voters
who identified themselves as likely to vote. Webb was the choice of
\(299\) of those surveyed. The article went on to state, ``Because
Webb's edge is equal to the margin error of plus or minus \(4\)
percentage point, it means that he can be considered slightly ahead''.

\begin{enumerate}
  \item The article stated that the margin error was $4\%$. Which is the significance level?
  \item Compute the confidence interval. Would you agree with that statement that Webb is slightly ahead?
  \item What is the interpretation of the margin error?
  \item Would you consider consecutive polls as samples from the same population? Justify your answer.
\end{enumerate}
\end{frame}

\begin{frame}{The article stated that the margin error was \(4\%\).
Which is the significance level?}
\protect\hypertarget{the-article-stated-that-the-margin-error-was-4.-which-is-the-significance-level}{}
\pause

The sample proportion is: \(\hat{p} = \frac{299}{597} = 0.5008\). Margin
of error for the sample proportion is given by the formula: \[
ME = z_{\alpha/2}\sqrt{\frac{\hat{p}(1-\hat{p})}{n}}
\] Hence

\[
0.04 = z_{\alpha/2}\sqrt{\frac{0.5008*(1-0.5008)}{597}}
\] \[
\Rightarrow z_{\alpha/2} \approx 1.9547
\]

Then, the significance level is: \begin{align*}
\alpha &= 2*P(Z>1.9547)\\ 
       &= 2*[1-P(Z\leq1.9547)]\\
       &\approx 0.05
\end{align*}
\end{frame}

\begin{frame}{Compute the confidence interval. Would you agree with that
statement that Webb is slightly ahead?}
\protect\hypertarget{compute-the-confidence-interval.-would-you-agree-with-that-statement-that-webb-is-slightly-ahead}{}
\pause

\(\hat{p}=299/597\), hence \(\hat{p}\pm 0.04\) gives the confidence
interval \((0.4608, 0.5408)\). No statistical reason to say Webb is
slightly ahead, interval symmetric around parity
(\(\hat{p}\approx 0.5\)).
\end{frame}

\begin{frame}{What is the interpretation of the margin error?}
\protect\hypertarget{what-is-the-interpretation-of-the-margin-error}{}
\pause

The margin error (ME) refers to the sample variability, that is it
reflects the extent to which \(\hat{p}\) varies if repeated sample of
the same size are drawn from the same population.
\end{frame}

\begin{frame}{Would you consider consecutive polls as samples from the
same population? Justify your answer.}
\protect\hypertarget{would-you-consider-consecutive-polls-as-samples-from-the-same-population-justify-your-answer.}{}
\pause

No, the opinions can change between one poll and the other (different
elections) because the opinion of the voting population can be affected
the scandals, reputation of the candidate, position of the candidates on
important events, etc.
\end{frame}

\begin{frame}{Exercise 2}
\protect\hypertarget{exercise-2}{}
A random sample of size 2, \(Y_1\) and \(Y_2\) is drawn from the the pdf
\[
f(y;\theta)=2y\theta^2,\qquad 0<y<\frac{1}{\theta}
\]

\begin{enumerate}
  \item What must $c$ be equal if the statistic 
$c(Y_1+2Y_2)$ to be an unbiased estimator of $1/\theta?$
  \item Find an alternative unbiased estimator for $1/\theta$.
\end{enumerate}
\end{frame}

\begin{frame}{What must \(c\) be equal if the statistic \(c(Y_1+2Y_2)\)
to be an unbiased estimator of \(1/\theta?\)}
\protect\hypertarget{what-must-c-be-equal-if-the-statistic-cy_12y_2-to-be-an-unbiased-estimator-of-1theta}{}
\pause

Note that \[
E(Y)=2\theta^2\int_{0}^{1/\theta}y^2dy=\frac{2\theta^2}{3}\left[y^3\right]_0^{1/\theta}=\frac{2}{3\theta}
\] By the linearity of the expectation \[
E\left[c(Y_1+2Y_2)\right]=c\left(\frac{2}{3\theta}+\frac{4}{3\theta}\right)=\frac{2c}{\theta}
\] Hence, \(c=1/2\).
\end{frame}

\begin{frame}{Find an alternative unbiased estimator for \(1/\theta\)}
\protect\hypertarget{find-an-alternative-unbiased-estimator-for-1theta}{}
\pause

Recalling that \(E(Y)=2/(3\theta)\), and unbiased estimator is \[
\frac{3}{4}\left(Y_1+Y_2\right)
\]
\end{frame}

\end{document}
