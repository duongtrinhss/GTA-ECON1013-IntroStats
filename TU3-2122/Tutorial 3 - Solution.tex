\documentclass[11pt,a4paper]{article}
%\documentclass[professionalfonts, 10pt]{beamer}
%\usetheme{default}
%\usetheme{Pittsburgh}
\usepackage{accessibility}
\usepackage[a4paper, total={6in, 8in}]{geometry}
\usepackage[pdflang={en-GB}]{hyperref}
\renewcommand{\familydefault}{\sfdefault}
%
%
\usepackage{enumitem}
\usepackage[T1]{fontenc}
\usepackage{hyperref}
\usepackage{kerkis}
\usepackage{stmaryrd}
%\usepackage{bm}
%\usepackage{mathrsfs}
\usepackage{bm}
\usepackage{amsmath,amssymb,amsfonts,amsthm,mathrsfs,dsfont}
%\usepackage{latexsym}
%\usefonttheme{professionalfonts}
%\usetheme{Copenhagen}
%\usecolortheme{beaver}
\usepackage[english]{babel}
\usepackage{textcomp}
\usepackage{marvosym}
\usepackage[latin1]{inputenc}
\usepackage{times}
\usepackage{textcomp}
%\usepackage{pifont}
%\usepackage{bbding}
%\usepackage{enumerate}
\usepackage{natbib}
%\usepackage{pgf,pgfarrows,pgfnodes,pgfautomata,pgfheaps}
\usepackage{color}
\usepackage{graphicx}
%
%
%\usepackage{authblk}
%\usepackage{MnSymbol}
\usepackage{semtrans}
\usepackage{units}
%\usepackage{mathrsfs}
%\usepackage{makeidx}
%
%
\newcommand\independent{\protect\mathpalette{\protect\independenT}{\perp}}

\def\independenT#1#2{\mathrel{\rlap{$#1#2$}\mkern2mu{#1#2}}}
\setcounter{MaxMatrixCols}{10}
\newcommand{\stackunder}[2]{\mathrel{\mathop{#2}\limits_{#1}}}
\newcommand{\fd}[1]{\Delta^{#1}_+}
%
\DeclareMathOperator{\E}{\mathbb{E}}
\DeclareMathOperator{\R}{\mathbb{R}}
\DeclareMathOperator{\Z}{\mathbb{Z}}
\DeclareMathOperator{\N}{\mathbb{N}}
\DeclareMathOperator{\var}{\mathbb{V}ar}
\DeclareMathOperator{\cov}{\mathbb{C}ov}
\DeclareMathOperator{\corr}{\mathbb{C}orr}
%
\def\vec{\mathop{\textmd{vec}}\nolimits}
\def\eig{\mathop{\textmd{eig}}\nolimits}
\def\tr{\mathop{\textmd{tr}}\nolimits}
\def\diag{\mathop{\textmd{diag}}\nolimits}
\def\vec{\mathop{\textmd{vec}}\nolimits}
\def\vh{\mathop{\textmd{vech}}\nolimits}
\def\berr{\begin{eqnarray*}}
\def\eerr{\end{eqnarray*}}
\def\ber{\begin{eqnarray}}
\def\eer{\end{eqnarray}}
\def\d{\textrm{d}}
\def\corr{\mathop{\textmd{Corr}}\nolimits}
%%%%%%%%%%%%%%%%%%%%%%%%%%%%%%%%%%%%%%%%%%%%%%%%%%%%%%%%%%%%%%%%%%%%%%%%%%%%%%%%%%
%%%%%%%%%%%%%%%%%%%%%%%%%%%%%%%%%%%%%%%%%%%%%%%%%%%%%%%%%%%%%%%%%%%%%%%%%%%%%%%%%%


\def\ln{\textrm{ln}}

\def\itemb{\item[$\bullet$]}
%%%%%%%%%%%%%%%%%%%%%%%%%%%%%%%%%%%%%%%%%%%%%%%%%%%%%%%%%%%%%%%%%%%%%%%%%%%%%%%%%%%%%
%%%%%%%%%%%%%%%%%%%%%%%%%%%%%%%%%%%%%%%%%%%%%%%%%%%%%%%%%%%%%%%%%%%%%%%%%%%%%%%%%%%%%






\newcommand{\V}[1]{``{#1}"}

\begin{document}
\begin{center}
\begin{Huge}
Tutorial 3
\end{Huge}
\end{center}


\begin{enumerate}[labelindent=0pt,labelwidth=0.75em,leftmargin=!]
\item In the mid-term elections of 2006, the political winds were shifting. One of the key races for control of the Senate was in Virginia, where challenger Jim Webb and incumbent George Allen were in a very tight race. Just a week before the election, the Associated Press reported on a CNN poll based on telephone interviews of 597 registered voters who identified themselves as likely to vote. Webb was the choice of 299 of those surveyed. The article went on to state, \V{Because Webb's edge is equal to the margin error of plus or minus 4 percentage point, it means that he can be considered slightly ahead}\footnote{Adapted from Lars and Morris (2014), \emph{Introduction to Mathematical Statistics and Its Applications}.}.
\begin{enumerate}
\item The article stated that the margin error was 4\%. Which is the significance level?
\item Compute the confidence interval. Would you agree with that statement that Webb is slightly ahead?
\item What is the interpretation of the margin error?
\item Would you consider consecutive polls as samples from the same population? Justify your answer.
\end{enumerate}
\item[]\textbf{Solution}
\begin{enumerate}
\item $0.04=c_{\alpha/2}\sqrt{\frac{0.25}{597}}$. Then $c_{\alpha/2}=1.9547$. $P(Z>1.9547)\simeq 0.975$.
\item $\hat{p}=299/597$, hence $\hat{p}\pm 0.04$ gives the interval $(0.4608, 0.5408)$. No statistical reason to say Webb is slightly ahead, interval symmetric around parity ($\hat{p}\simeq 0.5$).
\item The ME refers to the sample variability, that is it reflects the extent to which $\hat{p}$ varies if repeated sample of the same size are drawn from the same population.
\item No, the opinions can change between one poll and the other (different elections) because the opinion of the voting population can be affected the scandals, reputation of the candidate, position of the candidates on important events, etc.
\end{enumerate}
\item 
A random sample of size 2, $Y_1$ and $Y_2$ is drawn from the the pdf 
$$
f(y;\theta)=2y\theta^2,\qquad 0<y<\frac{1}{\theta}
$$
\begin{enumerate}
\item What must $c$ be equal if the statistic 
$$
c(Y_1+2Y_2)
$$
to be an unbiased estimator of $1/\theta?$
\item Find an alternative unbiased estimator for $1/\theta$.
\end{enumerate}
\item[]\textbf{Solution}
\begin{enumerate}
\item 
Note that
$$
\E(Y)=2\theta^2\int_{0}^{1/\theta}y^2dy=\frac{2\theta^2}{3}\left[y^3\right]_0^{1/\theta}=\frac{2}{3\theta}
$$
By the linearity of the expectation
$$
\E\left[c(Y_1+2Y_2)\right]=c\left(\frac{2}{3\theta}+\frac{4}{3\theta}\right)=\frac{2c}{\theta}
$$
Hence, $c=1/2$.
\item Recalling that $\E(Y)=2/(3\theta)$, and unbiased estimator is 
$$
\frac{3}{4}\left(Y_1+Y_2\right)
$$
\end{enumerate}
\end{enumerate}
\end{document}


\item As part of a \V{Math for the Twenty-First Century} initiative, Bayview High  was chosen to participate in the evaluation of a new algebra and geometry curriculum. In the  recent past, Bayview's students were considered \V{typical}, having earned scores on standardized exams that were very consistent with national averages.

Two years ago, a cohort of eighty-six Bayview students, all randomly selected, were assigned to a special set of classes that integrated algebra and geometry. According to test results that have just been released, those students averaged 502 on the math exam; nationwide seniors averaged 494 with a standard deviation of 124.
\begin{enumerate}
\item Can it be claimed at $5\%$ significance level that the new curriculum had effect?
\item Compute the p-value associated with the test statistics. How should it be interpreted?
\end{enumerate}
\end{enumerate}
\end{document}